%
% Template for DCS course projects
%
\documentclass[a4paper,11pt,oneside]{book}
\usepackage[latin1]{inputenc}
\usepackage[english]{babel}
\usepackage{amsfonts}
\usepackage{amsmath}
\usepackage{amssymb,amsmath,color}
\usepackage{cite}
\usepackage{graphicx}
\usepackage{float}

\begin{document}
\pagestyle{myheadings}

%%%%%%%%%%% Cover %%%%%%%%%%%
\thispagestyle{empty}                                                 
\begin{center}                                                            
    \vspace{5mm}
    {\LARGE UNIVERSIT\`A DI BOLOGNA} \\                       
      \vspace{5mm}
\end{center}
\begin{center}
  \includegraphics[scale=.27]{figs/logo_unibo}
\end{center}
\begin{center}
      \vspace{5mm}
      {\LARGE School of Engineering} \\
        \vspace{3mm}
      {\Large Master Degree in Automation Engineering} \\
      \vspace{20mm}
      {\LARGE Distributed Control Systems} \\
      \vspace{5mm}{\Large\textbf{COVERAGE CONTROL FOR COOPERATIVE MULTI-ROBOT NETWORKS}}                  
      \vspace{15mm}
\end{center}
\begin{flushleft}                                                                              
     {\large Professor: \textbf{\@ Giuseppe Notarstefano}} \\        
      \vspace{13mm}
\end{flushleft}
\begin{flushright}
      {\large Students:\\
      		\textbf{Donato Brusamento\\
      				Mattia Micozzi\\
      				Guido Carnevale\\
      				Lorenzo Draghetti}}\\
\end{flushright}        %capoverso allineato a destra
\begin{center}
\vfill
      {\large Academic year \@2018/2019} \\
\end{center}


\newpage
\thispagestyle{empty}

%%%%%% ABSTRACT %%%%%%%%%%
\begin{center}
\chapter*{}
\thispagestyle{empty}
{\Huge \textbf{Abstract}}\\
\vspace{15mm}
\end{center}
Put here the abstract of the project. 

%%%%%%%%%%%%%%%%%%%%%%%%%%%

\tableofcontents \thispagestyle{empty}
\listoffigures\thispagestyle{empty}

%%%%%% INTRODUZIONE %%%%%%%%%%
\chapter*{Introduction}
\addcontentsline{toc}{chapter}{Introduction}
A mobile wireless sensor network (MWSN) can be defined as a wireless sensor network (WSN) in which the sensor nodes are mobile. For example the nodes can be wheeled robots scattered in a given area.\\MWSNs are a an emerging field of research in contrast to their well-established predecessor. They are much more flexible than static sensor networks as they can be deployed in any scenario and deal with rapid topology changes due to node failures or new added sensors. In general each node consists of a radio transceiver, a microcontroller powered by a battery and one or more sensors to detect certain properties of the environment.\\
Typical applications of this kind of network are environment monitoring, surveillance, search and recovery operation, exploration.\\
A problem that can arise in this context is the optimization of the coverage of a given area knowing the distribution density function, defined on that area, of a given property we would like to measure. The objective of this project was to implement a distributed and asynchronous algorithm to solve this problem, following the method proposed in the research paper \cite{MR-GB:11}.\\
The framework (scenario?) we considered, as suggested on the abovementioned paper, is the following: n mobile sensor-robots modelled as simple integrators strewn on a convex polytope(area) defined in $\mathbb{R}^2$; on this area a random normal density function $\phi:Q\rightarrow \mathbb{R}_+$ is defined to represent the distribution of the feature that sensors have to measure. Each node has sensing(can locate the other nodes positions) and communication capabilities.  
The proposed algorithm is the Lloyd one, a gradient descent method that guarantees the convergence to an optimal solution from the point of view of nodes coverage and degradation of sensing capabilities exploiting the notion of Voronoi partition. Considering a convex polytope $Q$ in $\mathbb{R}^N$ and n sensors, for the sensor positions $P=\{p_1,...,p_n\}$ the optimal partition of $Q$ is the Voronoi partition  $V(P)=\{V_1,...,V_n\}$, where $V_i$ is the Voronoi cell of the i-th sensor: $$V_i=\{q\in Q|\> \|q-p_i\|\le\|q-p_j\|, \forall j\not= i\}$$\\
Basically each $V_i$ represents the space region in which sensor i is performing his task.\\
Conceptually this algorithm runs iteratively other 2 algorithm: one, called "Adjust-sensing radius algorithm", to compute the voronoi cell of each sensor and the other "Monitoring algorithm" to check if the computation of the Voronoi cell has to be updated because of some changes in the network(variation of nodes positions, node failures,...).\\
We implemented this procedure on Matlab in 2 ways:
\begin{enumerate}
	\item in a centralized sequential way
	\item in a distributed parallel fashion, exploiting files reading-writing to realize communication among nodes
\end{enumerate}
These two implementations and the differences between them are well-explained and analysed in this report\\
In particular, in the first chapter the problem set-up and the implemented solutions are described; the second chapter contains the analysis of the obtained results; final conclusions end the report.\\


 


 
\section*{Motivations}
\section*{Contributions}
Describe what is the contribution of this project.

% \section*{Organization}
%%%%%%%%%%%%%%%%%%%%%%%%%%%%%%%

%%%%%%%%% CAPITOLO  %%%%%%%%%%%%%%%%
\chapter{Chapter 1 title}
First-chapter for problem set-up and description of the implemented solution. 

\section{Section title}
\subsection {Subsection title}

Citation \cite{MR-GB:11}


%%%%%%%%% CAPITOLO  %%%%%%%%%%%%%%%%
\chapter{Chapter title}
Second chapter for description of the results (simulations and experiments where
applicable).
\section{Section title}
\subsection {Subsection title}

Citation \cite{MR-GB:11}


%%%%%%%%%%%%%%%%%%%%%%%%%%%%%%%%%%%%%

%%%%% SVILUPPI FUTURI %%%%%%
\chapter*{Conclusions} % and future developments}
\addcontentsline{toc}{chapter}{Conclusions} %  and future developments}
%%%%%%%%%%%%%%%%%%%%%%%%%%

% %%%% APPENDIX %%%%%
% \appendix
% \chapter{Appendix title}
% %%%%%%%%%%%%%%%%%%%%

%%%%%%%%%% BIBLIOGRAPHY %%%%%%%%%%%%%%
\bibliography{bibliography}{}
\bibliographystyle{plain}
\addcontentsline{toc}{chapter}{Bibliography}
%%%%%%%%%%%%%%%%%%%%%%%%%%%%%%%%%%%%%%

\end{document}
